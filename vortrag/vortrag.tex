%
% THE BEER-WARE LICENSE (Rev. 42):
% Ronny Bergmann <bergmann@math.uni-luebeck.de> wrote this file. As long as you
% retain this notice you can do whatever you want with this stuff. If we meet
% some day, and you think this stuff is worth it, you can buy me a beer or
% coffee in return.
%
% This file is just to get started - You need the corresponding Logo
%
\documentclass[german,10pt,xcolor=colortbl,compress
%,draft
]{beamer}
\usepackage{xunicode}
%\usepackage[T1]{fontenc}
\usepackage{calc}
\usepackage[ngerman]{babel} % Neue Rechtschreibung
\usepackage{amsmath,amsthm,amssymb,euscript} % AMS-LaTeX  
\usepackage{enumerate,graphicx}
\usepackage{makecell}
\usepackage{tikz}
\usetikzlibrary{patterns}

% Load Them
\usetheme[slogan=false, navigation=true, myriad=false]{UzL}
%
\setbeamertemplate{navigation symbols}{}
\title{One-Location Rowhammer Angriff}
% \subtitle{Eines Vortrages im neuen Design mit \LaTeX}
\date[]{2. Februar 2018\\[1ex] Aktuelle Themen IT-Sicherheit und Zuverlässigkeit}
\author[]{Gilian Henke, Dominik Mairhöfer}
% Clear Logo 1 to make the head smaller
%\institute[Universität zu Lübeck]{Institut für Mathematik\\Universität zu Lübeck}
%\clearlogo{1}
\setlogo{1}{.25\paperwidth}{Bilder/Logo_Uni_Luebeck_300dpi.png}%

\begin{document}
\section{Einführung}
\maketitle


\begin{frame}{Inhalt}
	\tableofcontents
\end{frame}

% TODO Motivation und Ziel
\begin{frame}{Motivation}

\begin{itemize}
	\item Neuer Rowhammer Angriff
	\item Neue Technik zum Ausnutzen des Rowhammer Angriffs
\end{itemize}
\pause
~\\
Wie gut funktioniert der Angriff?
~\\~\\
\begin{itemize}
	\item Durchführbarkeit des Angriffs testen
	\begin{itemize}
		\item schnell, zuverlässig, unauffällig, \dots
		\item Voraussetzungen
		\item Gegenmaßnahmen
	\end{itemize}
\end{itemize}

\end{frame}

% TODO Überblick über Angriff
\begin{frame}{Der Angriff}

\begin{itemize}
	\item Ziel: Verändern einer ausführbaren Datei ohne Schreibrechte
	\begin{itemize}
		\item z.B. \texttt{sudo} Datei für lokale Rechteausweitung
	\end{itemize}
	\item Basieren auf drei Teilen
	\begin{itemize}
		\item One-Location Rowhammer\\
		$ \Rightarrow $ Bitflips im Speicher erzeugen
		
		\item Memory Waylaing\\
		$ \Rightarrow $ Dateien an bestimmten stellen im Speicher platzieren
		
		\item Prefetch Side-Channel Angriff\\
		$ \Rightarrow $ Virtuelle Adressen in physikalische auflösen
	\end{itemize}
\end{itemize}

\end{frame}

\begin{frame}{Der Angriff - Ablauf}

\begin{enumerate}
	\item One-Location Rowhammer\\
	$ \Rightarrow $ Finde virtuelle Adresse bei der Bitflip möglich ist\\~\\
	\pause
	\item Prefetch Side-Channel Angriff\\
	$ \Rightarrow $ Finde physikalische Adressen zu dieser\\~\\
	\pause
	\item Memory Waylaying \& Prefetch Side-Channel Angriff\\
	$ \Rightarrow $ Platziere ausführbare Datei an dieser physikalischen Adresse\\~\\
	\pause
	\item One-Location Rowhammer\\
	$ \Rightarrow $ Erzeuge erneut Bitflip an gleicher physikalischer Adresse
	
\end{enumerate}

\end{frame}


\section{Hintergrund}
\subsection{One-Location Rowhammer}
% TODO  One-Location Rowhammer
\begin{frame}{One-Location Rowhammer}

\end{frame}

\subsection{Memory Waylaying}
% TODO Memory Waylaing
\begin{frame}{Memory Waylaying}
\begin{center}
	
	\begin{tikzpicture}
	
	\draw[step=0.5cm,very thin] (0,0) grid (5,5);
	
	% \draw[pattern=north west lines, pattern color=red] (0,0) rectangle (5,5);
	
	% Used space
	\fill[gray] (0,0) rectangle (0.5,0.5);
	\fill[gray] (1,4) rectangle (1.5,4.5);
	\fill[gray] (1,3) rectangle (5,3.5);
	\fill[gray] (0,2.5) rectangle (2,3);
	\fill[gray] (1,0) rectangle (4.5,0.5);
	\fill[gray] (0.5,1) rectangle (2,1.5);
	\fill[gray] (4,4.5) rectangle (4.5,5);
	\fill[gray] (4,4) rectangle (5,4.5);
	\fill[gray, opacity=1] (3.5,1.5) rectangle (5,2);
	
	% Target
	\draw[red!40,thin] (3,2) -- (3.5,2) -- (3.5,2.5) -- (3,2.5) -- (3,2);
	\fill[red!40] (3,2) rectangle (3.5,2.5);
	
	% binary
	\fill[white] (2,4) rectangle (2.5,4.5);
	\draw[thin] (2.1,4.1) -- (2.4,4.4);
	\draw[thin] (2.4,4.1) -- (2.1,4.4);
	
	% org binary
	\draw[green!40,thin] (0,1.5) -- (0.5,1.5) -- (0.5,2) -- (0,2) -- (0,1.5);
	\fill[green!40] (0,1.5) rectangle (0.5,2);
	
	% Legende
	\filldraw[gray, label=0:end] (6,5) rectangle (6.5,5.5);
	\node[right] at (6.7,5.25) {benutzter Speicher};
	
	\draw[red!40,thin] (6,4) -- (6.5,4) -- (6.5,4.5) -- (6,4.5) -- (6,4);
	\fill[red!40] (6,4) rectangle (6.5,4.5);
	\node[right] at (6.7,4.25) {Ziel};
	
	\draw[green!40,thin] (6,3) -- (6.5,3) -- (6.5,3.5) -- (6,3.5) -- (6,3);
	\fill[green!40] (6,3) rectangle (6.5,3.5);
	\node[right] at (6.7,3.25) {original Binary};
	
	\draw[thin] (6.1,2.1) -- (6.4,2.4);
	\draw[thin] (6.4,2.1) -- (6.1,2.4);
	\node[right] at (6.7,2.25) {geladene Binary};
	
	\draw[step=0.5cm,very thin] (0,0) grid (5,5);
	
	\end{tikzpicture}
	
\end{center}

%This is slide number \only<1>{1a}\only<2>{2}\only<3>{3}\only<4>{4}\only<5>{5}.
\end{frame}

\subsection{Prefetch Side-Channel Angriff}
% TODO Prefetch Side-Channel Angriff
\begin{frame}{Prefetch Side-Channel Angriff}

\end{frame}

\section{Ergebnisse}
% TODO Ergbnisse
\begin{frame}{Ergebnisse - One-Location Rowhammer}

\end{frame}
\begin{frame}{Ergebnisse - Memory Waylaing}

\end{frame}
\begin{frame}{Ergebnisse - Prefetch Side-Channel Angriff}

\end{frame}

\section{Fazit}
% TODO Fazit
\begin{frame}{Fazit}

\end{frame}
	
	
\end{document}
