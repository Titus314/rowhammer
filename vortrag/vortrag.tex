%
% THE BEER-WARE LICENSE (Rev. 42):
% Ronny Bergmann <bergmann@math.uni-luebeck.de> wrote this file. As long as you
% retain this notice you can do whatever you want with this stuff. If we meet
% some day, and you think this stuff is worth it, you can buy me a beer or
% coffee in return.
%
% This file is just to get started - You need the corresponding Logo
%
\documentclass[german,10pt,xcolor=colortbl,compress
%,draft
]{beamer}
\usepackage{xunicode}
%\usepackage[T1]{fontenc}
\usepackage{calc}
\usepackage[ngerman]{babel} % Neue Rechtschreibung
\usepackage{amsmath,amsthm,amssymb,euscript} % AMS-LaTeX  
\usepackage{enumerate,graphicx}
\usepackage{makecell}

% Load Them
\usetheme[slogan=false, navigation=true, myriad=false]{UzL}
%
\setbeamertemplate{navigation symbols}{}
\title{One-Location Rowhammer Angriff}
% \subtitle{Eines Vortrages im neuen Design mit \LaTeX}
\date[]{2. Februar 2018\\[1ex] Aktuelle Themen IT-Sicherheit und Zuverlässigkeit}
\author[]{Gilian Henke, Dominik Mairhöfer}
% Clear Logo 1 to make the head smaller
%\institute[Universität zu Lübeck]{Institut für Mathematik\\Universität zu Lübeck}
%\clearlogo{1}
\setlogo{1}{.25\paperwidth}{Bilder/Logo_Uni_Luebeck_300dpi.png}%

\begin{document}
\section{Einführung}
\maketitle


\begin{frame}{Inhalt}
	\tableofcontents
\end{frame}

% TODO Motivation und Ziel
\begin{frame}{Motivation}

\begin{itemize}
	\item Neuer Rowhammer Angriff
	\item Neue Technik zum Ausnutzen des Rowhammer Angriffs
\end{itemize}
\pause
~\\
Wie gut funktioniert der Angriff?
~\\~\\
\begin{itemize}
	\item Durchführbarkeit des Angriffs testen
	\begin{itemize}
		\item schnell, zuverlässig, unauffällig, \dots
		\item Voraussetzungen
		\item Gegenmaßnahmen
	\end{itemize}
\end{itemize}

\end{frame}

% TODO Überblick über Angriff
\begin{frame}{Der Angriff}

\begin{itemize}
	\item Ziel: Verändern einer ausführbaren Datei ohne Schreibrechte
	\begin{itemize}
		\item z.B. \texttt{sudo} Datei für lokale Rechteausweitung
	\end{itemize}
	\item Basieren auf drei Teilen
	\begin{itemize}
		\item One-Location Rowhammer\\
		$ \Rightarrow $ Bitflips im Speicher erzeugen
		
		\item Memory Waylaing\\
		$ \Rightarrow $ Dateien an bestimmten stellen im Speicher platzieren
		
		\item Prefetch Side-Channel Angriff\\
		$ \Rightarrow $ Virtuelle Adressen in physikalische auflösen
	\end{itemize}
\end{itemize}

\end{frame}

\begin{frame}{Der Angriff - Ablauf}

\begin{enumerate}
	\item One-Location Rowhammer\\
	$ \Rightarrow $ Finde virtuelle Adresse bei der Bitflip möglich ist\\~\\
	\pause
	\item Prefetch Side-Channel Angriff\\
	$ \Rightarrow $ Finde physikalische Adressen zu dieser\\~\\
	\pause
	\item Memory Waylaying \& Prefetch Side-Channel Angriff\\
	$ \Rightarrow $ Platziere ausführbare Datei an dieser physikalischen Adresse\\~\\
	\pause
	\item One-Location Rowhammer\\
	$ \Rightarrow $ Erzeuge erneut Bitflip an gleicher physikalischer Adresse
	
\end{enumerate}

\end{frame}


\section{Hintergrund}
\subsection{One-Location Rowhammer}
% TODO  One-Location Rowhammer
\begin{frame}{One-Location Rowhammer}

\end{frame}

\subsection{Memory Waylaing}
% TODO Memory Waylaing
\begin{frame}{Memory Waylaing}
%This is slide number \only<1>{1a}\only<2>{2}\only<3>{3}\only<4>{4}\only<5>{5}.
\end{frame}

\subsection{Prefetch Side-Channel Angriff}
% TODO Prefetch Side-Channel Angriff
\begin{frame}{Prefetch Side-Channel Angriff}

\end{frame}

\section{Ergebnisse}
% TODO Ergbnisse
\begin{frame}{Ergebnisse - One-Location Rowhammer}

\end{frame}
\begin{frame}{Ergebnisse - Memory Waylaing}

\end{frame}
\begin{frame}{Ergebnisse - Prefetch Side-Channel Angriff}

\end{frame}

\section{Fazit}
% TODO Fazit
\begin{frame}{Fazit}

\end{frame}
	
	
\end{document}
