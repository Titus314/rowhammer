Wie bei den Ergebnissen der einzelnen Tests zu sehen ist, funktioniert nicht jeder Teil des Angriffs problemlos und unter allen Bedingungen. Die Tests für den Rowhammer Angriff ergaben, dass ein One-Location Rowhammer auf unserer Hardware auch bei langen Ausführungen keine Bitflips erzeugte. Dies kann selbstverständlich an der eingesetzten Hardware liegen, jedoch ist es wahrscheinlich, dass auch auf anderer Hardware nicht schnell viele Bitflips gefunden werden. Der Vorteil des Rowhammer Teils ist jedoch, dass an dieser Stelle sehr einfach andere Rowhammer Varianten genutzt werden können. Wird beispielsweise die Variante Double-Sided Rowhammer genutzt, so ist es möglich und auch praktikabel diesen Angriffsteil durchzuführen. Keine der verwendeten Rowhammer Angriffe ist wirklich unauffällig, was jedoch der nicht Benutzung der Intel SGX Erweiterung geschuldet ist.

Das Memory Waylaying hingegen funktionierte wiederholbar, schnell und zuverlässig. Dieser Teil des Angriffs kann ohne Probleme auf jeder Hardware durchgeführt werden. Weiterhin hat sich das Memory Waylaying auch als sehr unauffällig herausgestellt, da es weder Arbeitsspeicher noch CPU stark beansprucht.

Die Prefetch Side-Channel Attacke war bei unseren Tests letztendlich nicht zu reproduzieren. Die genauen Gründe weswegen der Angriff nicht funktioniert hat, konnten nicht ermittelt und somit auch nicht behoben werden. Eine mögliche Ursache kann die verwendete Hardware sein. Entscheidend ist jedoch hierbei, dass dieser Teil des Angriffs nicht einfach ausgetauscht werden kann. Ohne die physikalische Adresse zu kennen, beziehungsweise zu wissen ob zwei virtuelle Adressen physikalische nebeneinander liegen, ist der Gesamtangriff nicht durchführbar.

Unter der Annahme es gäbe ein funktionierendes Orakel, dass physikalische Adressen zu virtuellen Adressen liefert, lassen sich folgende Aussagen über einen praktischen Einsatz des Gesamtangriffs machen:

Ein Angriff dieser Art wäre wahrscheinlich lediglich auf nicht überwachten Systemen durchführbar. Auf Servern, bei denen oft ECC Speicher im Einsatz ist, ist diese Art des Rowhammer Angriffs nicht möglich, da Bitflips erkannt und behoben werden können. Weiter würde eine einfache Überwachung der Auslastung von CPU und Arbeitsspeicher bei dem Angriff Auffälligkeiten entstehen lassen. Auch eine Benutzung des Systems während eines Angriff könnte bereits dafür sorgen, dass der Nutzer den Angriff aufgrund von Leistungseinbußen bemerkt.

In einem Szenario, bei welchem sich root Rechte auf einem System ohne ECC Speicher, das nicht überwacht wird und das während des Angriffs nicht in Benutzung ist, ist dieser Angriff durchaus praktikabel.

Als Schutz gegen den Angriff kommen mehrere Maßnahmen in Betracht. Einerseits gibt es die Möglichkeit, durch bestimmte Hardware den Angriff zu verhindern, beispielsweise durch den Einsatz von ECC Speicher. Da jedoch viele Endverbraucher, auf Grund von Nichtunterstützung, keinen ECC Speicher verwenden können, schützt dies nicht jeden. Eine weitere Möglichkeit wäre explizit Arbeitsspeicher zu verwenden, bei welchem bekanntermaßen nur sehr wenige Bitflips auftreten. Allein dies kann dafür sorgen, dass der Angriff nicht mehr in praktikabel durchführbar ist.

Auch ein Schutz in Form von Softwarepatches ist möglich. Auch wenn die Prefetch Side-Channel Attacke bei uns nicht funktionierte, wird sie auf jeden Fall durch Benutzung eines aktuellen Linux Kernels unmöglich und somit auch der Gesamtangriff.